% XeLaTeX 

\documentclass{minimal}
\usepackage[voc]{arabxetex} % default option is "novoc"
                       % other options are "voc", "fullvoc", "utf" and "trans"
% arabxetex loads bidi.sty and fontspec.sty		       

\usepackage{bidipoem}

\setmainfont{Linux Libertine}

%% \makeatletter
%% \def\Allah{\begingroup\allahfont\char"FDF2\endgroup}
%% \makeatother

\linespread{5}

\listfiles
 
\newfontfamily\arabicfont[Script=Arabic,Scale=2]{Amiri}

%% \renewcommand\baytwidth{0.25\textwidth}

\newcommand{\w}{\char"0627\char"0610}

\newcommand{\bayt}[2]{#1&\quad#2\\\textLR{\textarab[trans]{#2}}&\quad\textLR{\textarab[trans]{#1}}\\}

% you can also define \farsifont, \uighurfont, etc if you want
\begin{document}

%% This is some text in English with Arabic insertion \textarab{`arabI}.

%% This is some text in English with Arabic insertion \textarab[utf]{عربي}.

%% This is some text in English with Arabic insertion \textarab[voc]{`arabI}.

%% \SetTranslitConvention{dmg}

%% bi-Asmi al-ll_ahi al-ra.hmAni al-ra.hImi

\begin{arab}

\center{BASMALA}

\begin{traditionalpoem}
\bayt{al-ll_ahu al-ll_ahu al-ll_ahu al-ll_ah}{al-ll_ahu al-ll_ah lA il_aha illA al-ll_ah}
\bayt{ata"InAka bi-al-faqri yA _dA al-.ginY}{fa-anta alla_d"I lam tazal mu.hsinA}
\bayt{wa `awadtanA kulla fa.dliN `asY}{yadUmu alla_d"I minka `awadtanA}
\bayt{masAkInuka al-^su`_du qad wulihU"A}{bi-.hubbika i_d huwa aq.sY al-munY}
\bayt{i_dA kunta fI kulli .hAliN ma`I}{fa-`an .hamli zAdI anA fI .ginY}
\bayt{ra'aynAka fI kulli amriN badA}{wa laysa min al-amri ^say'uN lanA}
\bayt{satartu \w smakum .gIraTaN hA anA}{umawwihu bi-al^sa`bi wa-almun.hanY}
\bayt{fa-mA a.haduN fI al-.ginY mi_tlukum}{wa fI al-faqri lA a.haduN mi_tlunA}
\bayt{fa-'antum huwa al-.haqqu lA .gayrakum}{fa-yA layta ^si`rI anA man anA}
\bayt{fa-yA rabbi .salli `alY al-mu.s.tafY}{.salATaN takUnu ^sifA'aN lanA}
\bayt{wa-yA rabbi .salli `alY al-mu.s.tafY}{.salATaN takUnu dawA'aN lanA}
\bayt{wa-yA rabbi .salli `alY al-mu.s.tafY}{.salATaN takUnu amAnaN lanA}
\end{traditionalpoem}

\center{BASMALA}

\begin{traditionalpoem}
\bayt{a^skI li-min yA a.hmad \quad yawm .tal`iti al-ma^shad}{wi al-anbiyA' ti^shad \quad innak rasUlu al-ll_ah}
\bayt{law lA a_hAfa al-lUm \quad wa a_h^sY `itAba al-qUm}{lA _hu_d .habIbI dUm \quad hayyim fI .hubbi al-ll_ah}
\bayt{aqsamtu bi-al-furqAn \quad wa bi-sUraTi al-ra.hmAn}{annI anA al-walhAn \quad bi-.hubbi ibni `abdi al-ll_ah}
\bayt{aqsamtu bi-al-a`rAf \quad wa bi-sUraTi al-a.hqAf}{yA wa.h^sati al-an.sAf \quad .sAbir li.hukmi al-ll_ah}
\bayt{aqsamtu bi-al-.hawAmIn \quad wa bi-sUraTi al-ta.hrIm}{annI a.sba.htu saqIm \quad fI .hubbi ibni `abdi al-ll_ah}
\bayt{jabInuka al-wa.d.dA.h \quad fAq .diyA'i al-mi.sbA.h}{bAn al-hawA' fa.d.dA.h \quad asbY al-`uqUl wa-al-ll_ah}
\bayt{yA nA`isa al-ajfAn \quad yA fAtina al-.guzlAn}{yA mu_hjila al-a.g.sAn \quad i`.tif ra`Aka al-ll_ah}
\bayt{yA kAmila al-zayni \quad yA qurraTu al-`ayni}{awfI lanA al-dayni \quad karramaN li-.haqqi al-ll_ah}
\bayt{yA kAmila al-aw.sAf \quad yA layyina al-a`.tAf}{qalbI li-na.hwaka .tAf \quad yA ibni `abdi al-ll_ah}
\bayt{ir.ham mu.hibbInak \quad `A^sU `alY dInak}{mA a.hlY sawAd `aynAk \quad hayyimtanA wa-al-ll_ah}
\bayt{ra'aytuhu fI al-nUm \quad nizlat dumU`I `Um}{yA nAs anA al-ma.hrUm \quad min .hubbi .habIbi al-ll_ah}
\bayt{ra'aytuhu badaruN \quad ya.dwI kamA al-qamaru}{afdIhi bi-al-`umari \quad .t_ah_a rasUla al-ll_ah}
\bayt{ra'aytuhu fI al-layl \quad ya_htir kamA al-qandIl}{yA nAs anA al-`alIl \quad fI .hubbihi wa-al-ll_ah}
\end{traditionalpoem}

\center{BASMALA}

\begin{traditionalpoem}
\bayt{al-.salATu `alayka yA nUra ^samsI}{`adada al-kA'inAti min kulli jinsI}
\bayt{yA rabI`aN bi-al-yumni aqbalta bu^srY}{lam tazal fIka ayaTuN hiya laka kubrY}
\bayt{hiya a`lY min laylaTi al-qadri qadrA}{yA lahA min lawayliTi _dAta unsI}
\bayt{mawliduN fIhi \w htuzza IwAnu kisrY}{wa ra'at ummuhu qu.sUraN li-bu.srY}
\bayt{wa .tuyUruN min al-yawAqIti .humrA}{wa atY al-waylu wa al-_tubUru li-fursI}
\bayt{a.hsanu al-mu.hsinIna fI al-.hasanAti}{waladu al-.tAhirIna wa al-.tAhirAti}
\bayt{.tAba li-l-.tayyibIna wa al-.tayyibAti}{fidA'uhu abI wa ummI wa nafsI}
\bayt{wa yanAdI yawma al-qa.dA'i wa .hukmih}{wa qad `ataqnA li-'ajli sayyidi qawmih}
\bayt{min la.zA kulla man tasammY bi-'ismih}{h_aka_dA .sa.h.ha `an ma^sAyi_hi darsI}
\bayt{yA ilAhI yA man biyadihi al-_hayru kulluh}{bi-al-rasUli alla_dI tabayyana fa.dluh}
\bayt{jud `alY jam`inA bimA anta ahluh}{na.hnu na_h^sY bimA janaynA bi-'amsI}
\bayt{yA rasUla al-'il_ahi yA _hayra hAdI}{yA nabiyya al-hudY `alayka \w `timAdI}
\bayt{kun ^safI`I fI .gurbatI wa \w nfirAdI}{sayyamA laylata \w tti.sAlI bi-ramsI}
\bayt{wa .salATu al-ll_ahi ma`a ta.hiyyati rabbI}{tuhdA li-lnabiyyi al-hA^simI wa .sa.hbih}
\bayt{mA tarannama .tA'ir min fawqa qudsI}{`adada al-kA'inAti min kulli jinsI}
\end{traditionalpoem}


\center{BASMALA}

\begin{traditionalpoem}
  \bayt{all_ah yA `a.zIm anta al-`a.zIm}{qad hammanA amruN `a.zIm}
  \bayt{wa kullu amriN hammanA}{yahUnu bi-\w smika yA `a.zIm}
  \bayt{yA rabbanA bi-al-fAti.haT}{wa bi-al-rijAli al-.sAli.haT}
  \bayt{ij`al umUrunA nAji.haT}{na.hnu wa kulli al-muslimIn}
\end{traditionalpoem}

\end{arab}



\end{document}

