% XeLaTeX 

\documentclass{minimal}
\usepackage[voc]{arabxetex} % default option is "novoc"
                       % other options are "voc", "fullvoc", "utf" and "trans"
% arabxetex loads bidi.sty and fontspec.sty		       

\usepackage{bidipoem}

\setmainfont{Linux Libertine}

%% \makeatletter
%% \def\Allah{\begingroup\allahfont\char"FDF2\endgroup}
%% \makeatother

\linespread{5}

\listfiles
 
\newfontfamily\arabicfont[Script=Arabic,Scale=2]{Amiri}

%% \renewcommand\baytwidth{0.25\textwidth}

\newcommand{\w}{\char"0627\char"0610}

% you can also define \farsifont, \uighurfont, etc if you want
\begin{document}

%% This is some text in English with Arabic insertion \textarab{`arabI}.

%% This is some text in English with Arabic insertion \textarab[utf]{عربي}.

%% This is some text in English with Arabic insertion \textarab[voc]{`arabI}.

%% \SetTranslitConvention{dmg}

%% bi-Asmi al-ll_ahi al-ra.hmAni al-ra.hImi

\begin{arab}

BASMALA

\begin{traditionalpoem}

al-ll_ahu al-ll_ahu al-ll_ahu al-ll_ah &\quad al-ll_ahu al-ll_ah lA il_aha illA al-ll_ah\\
ata"InAka bi-al-faqri yA _dA al-.ginY &\quad fa-anta alla_d"I lam tazal mu.hsinA\\
wa `awadtanA kulla fa.dliN `asY &\quad yadUmu alla_d"I minka `awadtanA\\
masAkInuka al-^su`_du qad wulihU"A &\quad bi-.hubbika i_d huwa aq.sY al-munY\\
i_dA kunta fI kulli .hAliN ma`I &\quad fa-`an .hamli zAdI anA fI .ginY\\
ra'aynAka fI kulli amriN badA &\quad wa laysa min al-amri ^say'uN lanA\\
satartu \w smakum .gIraTaN hA anA &\quad umawwihu bi-al^sa`bi wa-almun.hanY\\
fa-mA a.haduN fI al-.ginY mi_tlukum &\quad wa fI al-faqri lA a.haduN mi_tlunA\\
fa-'antum huwa al-.haqqu lA .gayrakum &\quad fa-yA layta ^si`rI anA man anA\\
fa-yA rabbi .salli `alY al-mu.s.tafY &\quad .salATaN takUnu ^sifA'aN lanA\\
wa-yA rabbi .salli `alY al-mu.s.tafY &\quad .salATaN takUnu dawA'aN lanA\\
wa-yA rabbi .salli `alY al-mu.s.tafY &\quad .salATaN takUnu amAnaN lanA
\end{traditionalpoem}

\end{arab}



\end{document}

