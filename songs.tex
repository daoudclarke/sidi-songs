% XeLaTeX 

\documentclass{minimal}
\usepackage[voc]{arabxetex} % default option is "novoc"
                       % other options are "voc", "fullvoc", "utf" and "trans"
% arabxetex loads bidi.sty and fontspec.sty		       

\usepackage{bidipoem}

\setmainfont{Linux Libertine}

%% \makeatletter
%% \def\Allah{\begingroup\allahfont\char"FDF2\endgroup}
%% \makeatother

\linespread{5}

\listfiles
 
\newfontfamily\arabicfont[Script=Arabic,Scale=2]{Amiri}

%% \renewcommand\baytwidth{0.25\textwidth}

\newcommand{\w}{\char"0627\char"0610}

% you can also define \farsifont, \uighurfont, etc if you want
\begin{document}

%% This is some text in English with Arabic insertion \textarab{`arabI}.

%% This is some text in English with Arabic insertion \textarab[utf]{عربي}.

%% This is some text in English with Arabic insertion \textarab[voc]{`arabI}.

%% \SetTranslitConvention{dmg}

%% bi-Asmi al-ll_ahi al-ra.hmAni al-ra.hImi

\begin{arab}

\center{BASMALA}

\begin{traditionalpoem}
al-ll_ahu al-ll_ahu al-ll_ahu al-ll_ah &\quad al-ll_ahu al-ll_ah lA il_aha illA al-ll_ah\\
ata"InAka bi-al-faqri yA _dA al-.ginY &\quad fa-anta alla_d"I lam tazal mu.hsinA\\
wa `awadtanA kulla fa.dliN `asY &\quad yadUmu alla_d"I minka `awadtanA\\
masAkInuka al-^su`_du qad wulihU"A &\quad bi-.hubbika i_d huwa aq.sY al-munY\\
i_dA kunta fI kulli .hAliN ma`I &\quad fa-`an .hamli zAdI anA fI .ginY\\
ra'aynAka fI kulli amriN badA &\quad wa laysa min al-amri ^say'uN lanA\\
satartu \w smakum .gIraTaN hA anA &\quad umawwihu bi-al^sa`bi wa-almun.hanY\\
fa-mA a.haduN fI al-.ginY mi_tlukum &\quad wa fI al-faqri lA a.haduN mi_tlunA\\
fa-'antum huwa al-.haqqu lA .gayrakum &\quad fa-yA layta ^si`rI anA man anA\\
fa-yA rabbi .salli `alY al-mu.s.tafY &\quad .salATaN takUnu ^sifA'aN lanA\\
wa-yA rabbi .salli `alY al-mu.s.tafY &\quad .salATaN takUnu dawA'aN lanA\\
wa-yA rabbi .salli `alY al-mu.s.tafY &\quad .salATaN takUnu amAnaN lanA
\end{traditionalpoem}

\center{BASMALA}

\begin{traditionalpoem}
a^skI li-min yA a.hmad \quad yawm tal`iti al-ma^shad &\quad wi al-anbiyA' ti^shad \quad innak rasUlu al-ll_ah\\
law lA a_hAfa al-lUm \quad wa a_h^sY `itAba al-qUm &\quad lA _hu_d .habIbI dUm \quad hayyim fI .hubbi al-ll_ah\\
aqsamtu bi-al-furqAn \quad wa bi-sUraTi al-ra.hmAn &\quad annI anA al-walhAn \quad bi-.hubbi ibni `abdi al-ll_ah\\
aqsamtu bi-al-a`rAf \quad wa bi-sUraTi al-a.hqAf &\quad yA wa.h^sati al-an.sAf \quad .sAbir li.hukmi al-ll_ah\\
aqsamtu bi-al-.hawAmIn \quad wa bi-sUraTi al-ta.hrIm &\quad annI a.sba.htu saqIm \quad fI .hubbi ibni `abdi al-ll_ah\\
jabInuka al-wa.d.dA.h \quad fAq .diyA'i al-mi.sbA.h &\quad bAn al-hawA' fa.d.dA.h \quad asbY al-`uqUl wa-al-ll_ah\\
yA nA`isa al-ajfAn \quad yA fAtina al-.guzlAn &\quad yA mu_hjila al-a.g.sAn \quad i`.tif ra`Aka al-ll_ah\\
yA kAmila al-zayni \quad yA qurraTu al-`ayni &\quad awfI lanA al-dayni \quad karramaN li-.haqqi al-ll_ah\\
yA kAmila al-aw.sAf \quad yA layyina al-a`.tAf &\quad qalbI li-na.hwaka .tAf \quad yA ibni `abdi al-ll_ah\\
ir.ham muhibbInAk \quad `A^sU `alY dInAk &\quad mA a.hlY sawAd `aynAk \quad hayyimtanA wa-al-ll_ah\\
ra'aytahu fI al-nUm \quad nizlat dumU`I `Um &\quad yA nAs anA al-ma.hrUm \quad min .hubbi .habIbi al-ll_ah\\
ra'aytuhu badaruN \quad ya.dwI kamA al-qamaru &\quad afdIhi bi-al-`umari \quad .t_ah_a rasUla al-ll_ah\\
ra'aytuhu fI al-layl \quad ya_htir kamA al-qandIl &\quad yA nAs anA al-`alIl \quad fI .hubbihi wa-al-ll_ah\\
\end{traditionalpoem}

\center{BASMALA}

\begin{traditionalpoem}
  al-.salATu `alayka yA nUra ^samsI &\quad `adada al-kA'inAti min kulli jinsI\\
  yA rabI`aN bi-al-yumni aqbalta bu^srY &\quad lam tazal fIka ayaTa hiya laka kubrY\\
  hiya a`lY min laylaTi al-qadri qadrA &\quad yA lahA min lawaylihi _dAta unsI\\
  mawliduN fIhi \w htuzza IwAnu kisrY &\quad wa ra'at ummuhu qu.sUraN li-bu.srY\\
  wa .tuyUruN min al-yawAqIti .humrA &\quad wa atY al-waylu wa al-_tubUru li-fursI\\
  a.hsanu al-mu.hsinIna fI al-.hasanAti &\quad waladu al-.tAhirIna wa al-.tAhirAti\\
  .tAba li-l-.tayyibIna wa al-.tayyibAti &\quad fidA'uhu abI wa ummI wa nafsI\\
  wa yanAdI yawma al-qa.dA'i wa .hukmih &\quad wa qad `ataqnA li-'ajli sayyidi qawmih\\
  min la.zA kulla man tasammY bi-'ismih &\quad h_aka_dA .sa.h.ha `an ma^sAyi_hi darsI\\
  yA ilAhI yA man biyadihi al-_hayru kulluh &\quad bi-al-rasUli alla_dI tabayyana fa.dluh\\
  jud `alY jam`inA bimA anta ahluh &\quad na.hnu na_h^sY bimA janaynA bi-'amsI\\
  yA rasUla al-'il_ahi yA _hayra hAdI &\quad yA nabiyya al-hudY `alayka \w `timAdI\\
  kun ^safI`I fI .gurbatI wa \w nfirAdI &\quad sayyamA laylata \w tti.sAlI bi-ramsI\\
  wa .salATu al-ll_ahi ma`a ta.hiyyati rabbI &\quad tuhdA li-lnabiyyi al-hA^simI wa .sa.hbih\\
  mA tarannama .tA'ir min fawqa qudsI &\quad `adada al-kA'inAti min kulli jinsI\\
\end{traditionalpoem}

\end{arab}



\end{document}

